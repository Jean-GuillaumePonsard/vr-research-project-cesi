\documentclass[12pt, openany, twocolumn]{article}
\usepackage[utf8]{inputenc}
\usepackage[top = 1.5cm, left = 1.5cm, right =1.5cm, bottom = 1.5cm]{geometry}
\usepackage[pdfborder ={0 0 0}]{hyperref}

\title{Virtual Reality for human rehabilitation and medical training}
\author{ Aurélia Besse, Etienne Duverney, Jean-Guillaume Ponsard, Romain Junca }
\date{March 2019}

\begin{document}

\maketitle

\section{Context} 
\hspace{0.5cm}
This document is presenting the articles and the context leading to the problematic for our research project about Virtual Reality for human rehabilitation.
\\

The Virtual Reality (VR) technology is an interactive computer-generated experience taking place in a simulated environment. 
\\
Nowadays, the VR technology is composed of a head mounted display and 3D audio headphones.
\\
This technology allows for a fully imersive experience for the user.
\\

Virtual Reality has been commercially available since the late 80's, with the first systems sold by VPL Research.
This technology has always evolved through time thanks to better computer technologies and better softwares.
\\
This contributed to the "rebirth" of the VR in the late 90's and later in the late 2010's.
\\


With the development of low-cost devices, this rehabilitation can be continued at home, easing the access to these tools, in addition to their ludic and thus motivating properties.
\\ Recent technological advances have led to considerable cost reductions for VR equipments, and several companies are selling headsets that consist of 2 lenses and a place to insert a smartphone for less than \$20.
\\ A relatively lowpriced virtual-reality-based training program would be a more effective and cheaper way to exercise than attending a class in sport center.
\\ The democratization of the technology also allows the telerehabilitation. This means a patient can be treated by professionals from all around the world. 
\\

Within Medicine, VR has been used in teaching anatomy, training in diagnostic procedures (such as virtual colonoscopy, or virtual bronchroscopy), teaching open and minimally-invasive surgery procedures, and in rehabilitation. \\
Virtual Reality system can provide multimodal stimuli, such as visual and auditory stimuli, and can also be used to evaluate the patient’s multimodal integration and to aid rehabilitation of cognitive abilities. 
VR is similar enough to reality to provide an effective training environment for rehabilitation. \\

In rehabilitation therapy, where repetitive feedback and motor learning are necessary, a virtual reality system can provide adequate motivation of such a mechanism \\
In the medical field, VR has been used for the training of surgeons, especially for laparoscopic surgery, or for the treatment of phobias for example. The secure environment allows to control the stimuli presented to the patient so he can face his fear gradually. \\
Also, recent reports have described the use of virtual reality (VR) as a method of distraction during
procedures such as administering vaccines or drawing blood. \\


\section{Problematic}

Regarding nowadays advances of Virtual Reality technologies for rehabilitation, how can we facilitate the use of these and make them more accessible and secure for both the medical staff and the patients ? 

\nocite{*}
\bibliographystyle{plain}
\bibliography{database}
\end{document}