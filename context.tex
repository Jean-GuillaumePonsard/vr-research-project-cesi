\documentclass[12pt, openany, twocolumn]{article}
\usepackage[utf8]{inputenc}
\usepackage[top = 1.5cm, left = 1.5cm, right =1.5cm, bottom = 1.5cm]{geometry}
\usepackage[pdfborder ={0 0 0}]{hyperref}

\title{In which way the use of virtual reality increases the efficiency of a learning process ?}
\author{ Aurélia Besse, Etienne Duverney, Jean-Guillaume Ponsard, Romain Junca }
\date{March 2019}

\begin{document}

\maketitle

\section{Context} 
\hspace{0.5cm} 
This document is presenting the articles and the context leading to the problematic for our research project about Virtual Reality for human rehabilitation and learning.
% Insert quote with the parallel between rehabilitation and learning
Past studies showed that learning and rehabilitation are linked thanks to muscular and memory training in both fields (medical and general). 
Indeed, learning systems based on immitation enhance motor learning in healthy and disabled individuals \cite{holdenUseVirtualEnvironments, bastianUnderstandingSensorimotorAdaptation2008b}.
Memory games have proved to be efficient in rehabilitating of brain functions lost in stroke accidents. From a past exprience, we can say patients improved their memory ability unexpectedly while enjoying playing the game. 

As we do not have the required medical knowledge, we will demonstrate our statements with casual exercices.  
These exercises train the same parts of brain and muscles as some rehabilitation trainings.
Thus we can admit the benefits brought up in the medical field too.
\\

The Virtual Reality (VR) technology is an interactive computer-generated experience taking place in a simulated environment. 
\\
Nowadays, the VR technology is composed of a head mounted display and 3D audio headphones.
\\
This technology allows a fully imersive experience for the user \cite{sveistrupMotorRehabilitationUsing2004}.
\\

Virtual Reality has been commercially available since the late 80's, with the first systems sold by VPL Research.
This technology has always evolved through time thanks to better computer technologies and better softwares.
\\
This contributed to the "rebirth" of the VR in the late 90's \cite{burdeaVirtualRehabilitationBenefits2003} and later in the late 2010's.
\\


With the development of low-cost devices, this rehabilitation can be continued at home, easing the access to these tools, in addition to their ludic and thus motivating properties.
Indeed, motivation plays a major role during the learning process, as it helps to get quick and better results \cite{kangBenefitRetrievalPractice2014, christophelRelationshipsTeacherImmediacy1990, kinzieRequirementsBenefitsEffective1990b}.
\\ Recent technological advances have led to considerable cost reductions for VR equipments, and several companies are selling headsets that consist of 2 lenses and a place to insert a smartphone for less than \$20 \cite{araneVirtualRealityPain2017}.
\\ A relatively lowpriced virtual-reality-based training program would be a more effective and cheaper way to exercise than attending a class in sport center \cite{kimEffectsVRbasedWii2014}.
\\ The democratization of the technology also allows the telerehabilitation \cite{burdeaVirtualRehabilitationBenefits2003}. This means a patient can be treated by professionals from all around the world. 
\\

Within Medicine, VR has been used in teaching anatomy, training in diagnostic procedures, in rehabilitation, teaching open and minimally-invasive surgery procedures.  \cite{burdeaVirtualRehabilitationBenefits2003}. \\
Virtual Reality system can provide multimodal stimuli, such as visual and auditory stimuli, and can also be used to evaluate the patient’s multimodal integration and to aid rehabilitation of cognitive abilities \cite{bioulacQuApportentOutils2018, morelAdvantagesLimitationsVirtual2015}. 
VR is similar enough to reality to provide an effective training environment for rehabilitation. \\

In rehabilitation therapy, where repetitive feedback and motor learning are necessary, a virtual reality system can provide adequate motivation of such a mechanism \cite{kimEffectsVRbasedWii2014}.\\
In the medical field, VR has been used for the training of surgeons \cite{laverVirtualRealityStroke2017} or for the treatment of phobias \cite{morelAdvantagesLimitationsVirtual2015}. The secure environment allows to control the stimuli presented to the patient so he can face his fear gradually \cite{morelAdvantagesLimitationsVirtual2015}. \\
Also, recent reports have described the use of virtual reality (VR) as a method of distraction during procedures such as administering vaccines or drawing blood \cite{araneVirtualRealityPain2017}. \\
\\

In multiple articles we learn that the comparison between rehabilitation with or without VR proved that patients using this technology were more motivated and showed high levels of compliance during the process of rehabilitation \cite{sampaioDoesVirtualRealitybased2016, chenProgressSensorimotorRehabilitative2014}. \\
We can also notice that these patients showed better results and progress during these tests \cite{corbettaRehabilitationThatIncorporates2015, saposnikEffectivenessVirtualReality2010, chenProgressSensorimotorRehabilitative2014, saposnikgustavoVirtualRealityStroke2011}. \\
For example, these studies suggests a 5 times higher growth of chances of improvement in motor strength for patients who experienced a stroke after using a VR system \cite{saposnikgustavoVirtualRealityStroke2011}.\\
However, there is still work to do because, despite the number of studies about the benefits of VR in medical rehabilitation, and the number of patient who used it, and even the improvment observed, it is still not enough to prove that this method is 100\% better than the usual methods \cite{saposnikEffectivenessVirtualReality2010, saposnikgustavoVirtualRealityStroke2011, luque-morenoDecadeProgressUsing2015}.\\

\section{Experiment}

During our experiment, we want to demonstrate multiples improvement related to this technology:

\begin{itemize}
    \item The efficiency of VR excercices that require memory, movement precision and speed
    \item The precision of the results implying an easiness of calculation of the different parameters (max level/reaction time/...)
\end{itemize}

\bibliographystyle{plain}
\bibliography{database}
\end{document}