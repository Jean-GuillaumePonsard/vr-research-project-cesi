\documentclass{article}
\usepackage[utf8]{inputenc}

\title{Virtual Reality for human rehabilitation}
\author{ Aurélia Besse, Etienne Duverney, Jean-Guillaume Ponsard, Romain Junca }
\date{March 2019}

\begin{document}

\maketitle

\section{Introduction}

This document is presenting the articles and the context leading to a problematic for our research project about Virtual Reality for human rehabilitation.


\section{Context}

Definition:
The Virtual Reality technology is an interactive computer-generated experience taking place in a simulated environment. Nowadays its generally composed of a head mounted display and 3D audio headphones. \\
This technology allows for a fully imersive experience for the user.

Evolution:


Accessibilité:
Allows the telerehabilitation, where the therapist is distant 
\\ With the development of low-cost devices, this rehabilitation can be continued at home, easing the access to these tools, in addition to their ludic and thus motivating properties
\\ Recent technological advances have led to considerable cost reductions for VR equipment, and several companies are selling headsets that consist of 2 lenses and a place to insert a smartphone for less than \$20.
\\ A relatively lowpriced virtual-reality-based training program would be a more effective and cheaper way to exercise than attending a class in sport center.

état de la VR aujourd'hui:


Domaine médical:


Is it Good ?


\section{Problematic}


\section{Summary of our articles}



\end{document}