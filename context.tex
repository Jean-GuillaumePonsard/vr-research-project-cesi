\documentclass{article}
\usepackage[utf8]{inputenc}

\title{Virtual Reality for human rehabilitation}
\author{ Aurélia Besse, Etienne Duverney, Jean-Guillaume Ponsard, Romain Junca }
\date{March 2019}

\begin{document}

\maketitle

\section{Introduction}

This document is presenting the articles and the context leading to a problematic for our research project about Virtual Reality for human rehabilitation.


\section{Context}

Definition:
The Virtual Reality technology is an interactive computer-generated experience taking place in a simulated environment. Nowadays its generally composed of a head mounted display and 3D audio headphones. \\
This technology allows for a fully imersive experience for the user.

Evolution:


Accessibilité:


état de la VR aujourd'hui: \\



Domaine médical: \\
Virtual Reality system can provide multimodal stimuli, such as visual and auditory stimuli, and can also be used to evaluate the patient’s multimodal integration and to aid rehabilitation of cognitive abilities. \\
Within Medicine, VR has been used in teaching anatomy, training in diagnostic procedures (such as virtual colonoscopy, or virtual bronchroscopy), teaching open and minimally-invasive surgery procedures, and in rehabilitation. \\
Virtual Reality system can provide multimodal stimuli, such as visual and auditory stimuli, and can also be used to evaluate the patient’s multimodal integration and to aid rehabilitation of cognitive abilities. \\

VR is similar enough to reality to provide an effective training environment for rehabilitation. \\


Is it Good ? \\
in rehabilitation therapy, where repetitive feedback and motor learning are necessary, a virtual reality system can provide adequate motivation of such a mechanism \\
In the medical field, VR has been used for the training of surgeons, especially for laparoscopic surgery, or for the treatment of phobias for example. The secure environment allows to control the stimuli presented to the patient so he can face his fear gradually. \\
Also, recent reports have described the use of virtual reality (VR) as a method of distraction during
procedures such as administering vaccines or drawing blood. \\


\section{Problematic}


\section{Summary of our articles}



\end{document}