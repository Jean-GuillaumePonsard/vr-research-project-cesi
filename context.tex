\documentclass{article}
\usepackage[utf8]{inputenc}

\title{Virtual Reality for human rehabilitation}
\author{ Aurélia Besse, Etienne Duverney, Jean-Guillaume Ponsard, Romain Junca }
\date{March 2019}

\begin{document}

\maketitle

\section{Introduction}

This document is presenting the articles and the context leading to a problematic for our research project about Virtual Reality for human rehabilitation.


\section{Context}

Definition:
The Virtual Reality (VR) technology is an interactive computer-generated experience taking place in a simulated environment. 
\\
Nowadays, the VR technology is composed of a head mounted display and 3D audio headphones.
\\
This technology allows for a fully imersive experience for the user.

Evolution:
Virtual Reality has been commercially available since the late 80's, with the first systems sold by VPL Research.
This technology has always evolved through time thanks to better computer technology and better softwares.
\\
This contributed to the "rebirth" of the VR in the late 90's and later in the late 2010's.
\\



Accessibilité:


état de la VR aujourd'hui: \\



Domaine médical: \\
Virtual Reality system can provide multimodal stimuli, such as visual and auditory stimuli, and can also be used to evaluate the patient’s multimodal integration and to aid rehabilitation of cognitive abilities. \\
Within Medicine, VR has been used in teaching anatomy, training in diagnostic procedures (such as virtual colonoscopy, or virtual bronchroscopy), teaching open and minimally-invasive surgery procedures, and in rehabilitation. \\
Virtual Reality system can provide multimodal stimuli, such as visual and auditory stimuli, and can also be used to evaluate the patient’s multimodal integration and to aid rehabilitation of cognitive abilities. \\

VR is similar enough to reality to provide an effective training environment for rehabilitation. 
Is it Good ?


\section{Problematic}


\section{Summary of our articles}



\end{document}